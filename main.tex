\documentclass[dvipsnames]{beamer}
\beamertemplatenavigationsymbolsempty
\usetheme{Boadilla}
\usefonttheme[onlymath]{serif}

\usepackage{amsmath}
\usepackage{bm}
\usepackage{bbm}
\usepackage{mathrsfs}
\usepackage{mathtools}
\usepackage[cal=boondoxo]{mathalpha}

% Change horizontal spacing
\setlength{\tabcolsep}{3pt}

\usepackage[none]{hyphenat} % no hyphenation

\usepackage{array}

\usepackage{cancel}

\usepackage[style=authoryear,maxcitenames=2,backend=biber,citetracker=true]{biblatex}
\addbibresource{references.bib}

\usepackage{verbatim}

\newcommand{\credit}[2]{\par\hfill \footnotesize #1 credit:~\itshape\citeauthor{#2} (\citeyear{#2})}
\renewcommand{\cite}[1]{(\citeauthor{#1}, \citeyear{#1})}
\newcommand{\citefoot}[1]{\citeauthor{#1} (\citeyear{#1})}
\newcommand{\matr}[1]{#1}

\newcommand{\red}[1]{{\color{red} #1}}

\title[XAI with Shapley Value]
{Explainable Machine Learning\\with Shapley Value}
%\subtitle{}
\author{Gianmarco Midena}
\institute{Aalto University}
\date{12 March 2024}

\begin{document}
\begin{frame}
\titlepage
\end{frame}

%\begin{frame}{Outline}
%\tableofcontents
%\end{frame}

\begin{frame}{The Attribution Problem\footnotemark}
	\begin{itemize}\setlength\itemsep{2em}
		\item<1-> Distribute the prediction score of a model to its input features for a specific data point.
		\item<2-> How each feature affects the prediction for a particular data point.
		\item<3-> Importance of a feature value to a prediction
		\item<4-> Attributions have explanatory value
		\item<5-> What-if analysis
	\end{itemize}
	\footnotetext[1]{\citefoot{ribeiro2016should}, \citefoot{lundberg2017unified}, \citefoot{sundararajan2017axiomatic}.}
\end{frame}

\begin{frame}{Example - Probability of Cervical Cancer for a Woman}
	\begin{center}
		\includegraphics[scale=0.6]{images/shapley-cervical-plot-1.jpeg}
	\end{center}
\credit{Image}{molnar2020interpretable}
\end{frame}

\begin{frame}{Risk Factors for Cervical Cancer\footnotemark}
\begin{itemize}
	\item Has patient ever had a sexually transmitted disease (STD) [binary]
	\item Number of sexual partners
	\item Number of STD diagnoses
	\item Number of pregnancies
	\item First sexual intercourse (age in years)
	\item Hormonal contraceptives (in years)
	\item Age in years
	\item Hormonal contraceptives [binary]
	\item Smokes (binary)
	\item Number of years with an intrauterine device (IUD)
	\item Intrauterine device (IUD) [binary]
	\item Time since first STD diagnosis
	\item Time since last STD diagnosis
	\item Smokes (in years)
\end{itemize}
\footnotetext[2]{\citeauthor{risk_factors_dataset} (\citeyear{risk_factors_dataset})}
\end{frame}

\begin{frame}{Example - Number of Rented Bikes for a Day}
	\begin{center}
		\includegraphics[scale=0.6]{images/shapley-bike-plot-1.jpeg}
	\end{center}
	\credit{Image}{molnar2020interpretable}
\end{frame}

\begin{frame}{Bike Rental Features\footnotemark}
	\begin{itemize}
		\item Temperature in degrees Celsius
		\item Season: spring, summer, fall or winter
		\item Working day or weekend
		\item Holiday or not
		\item Wind speed in km per hour
		\item Year: 2011 or 2012
		\item Nr. days since 01.01.2011 (the first day in the dataset).
		\item Weather situation:
		\begin{enumerate}[a]
			\item clear, few clouds, partly cloudy, cloudy
			\item mist + clouds, mist + broken clouds, mist + few clouds, mist
			\item light snow, light rain + thunderstorm + scattered clouds, light rain + scattered clouds
			\item heavy rain + ice pallets + thunderstorm + mist, snow + mist
		\end{enumerate}
		\item Relative humidity percentage
	\end{itemize}
	\footnotetext[3]{\citeauthor{bike_sharing_dataset} (\citeyear{bike_sharing_dataset})}
\end{frame}

\begin{frame}{Linear Model}
	\begin{itemize}\setlength\itemsep{2em}
		\item<1-> Prediction function
		\begin{equation}
			\hat{f}(\bm{x})=w_0+w_1 x_{1} + \dots + w_p x_ p
		\end{equation}
		\begin{itemize}\setlength\itemsep{0.5em}
			\item $x_j$: value of feature $j$
			\item $w_j$: weight corresponding to feature $j$
			\begin{itemize}\setlength\itemsep{0.5em}
				\item $j$-th feature \textbf{global} importance 
				\\-Standardized input features
				\\-(typically) \red{How a \textbf{specific} feature value influences the prediction\\is more interesting!}
			\end{itemize}
			\item $p$: nr. features
		\end{itemize}
		\item<2-> No feature interaction $\rightarrow$ \emph{individual effects} are easy to compute
	\end{itemize}
\end{frame}
\begin{frame}[shrink=20]{Linear Model - Feature Contribution}
	\begin{align}
		\begin{split}
			\phi^{add}_j\left(\bm{x};\hat{f}\right)
			&=w_{j}x_j-\mathbb{E}\left[w_{j}X_{j}\right]\\
			&=w_{j}\left(x_j-\mathbb{E}\left[X_{j}\right]\right)\\
			&=\hat{f}(\bm{x}) - \mathbb{E}\left[\hat{f}(x_1, \dots, X_{j}, \dots, x_n)\right]
		\end{split}
	\end{align}
	\begin{itemize}\setlength\itemsep{0.5em}
		\item[]
		where
		\begin{itemize}
			\item $\mathbb{E}(w_{j}X_{j})$: mean effect estimate for feature $j$
		\end{itemize}
		\vspace{0.5em}
		\item<1-> $j$-th feature contribution
		\item<2-> $j$-th feature effect minus average $j$-th feature effect
		\item<3-> Contribution of feature $X_j$ with value $x_j$
		\\minus the expected contribution of feature $X_j$
		\item<4-> Model prediction
		\\minus expected prediction if $j$-th feature value is not known
		\item<5-> Perturbs the value of the $j$-th feature only
		\item<6-> Keeps the value of all other features
		\item<7-> Independent w.r.t. the values of all the features but $j$
		\item<8-> Situational importance of $X_j = x_j$~\cite{achen1982interpreting}
	\end{itemize}
\end{frame}
\begin{frame}[shrink=20]{Linear Model - Total Feature Contribution}
	\begin{align}
		\begin{split}
			\sum_{j=1}^{p}\phi^{add}_j\left(\bm{x};\hat{f}\right) =&\sum_{j=1}^p\bigl(w_{j}x_j-\mathbb{E}\left[w_{j}X_{j}\right]\bigr)\\
			=&\left(w_0+\sum_{j=1}^pw_{j}x_j\right)-\left(w_0+\sum_{j=1}^{p}\mathbb{E}\left[w_{j}X_{j}\right]\right)\\
			=&\hat{f}(\bm{x})-\mathbb{E}\left[\hat{f}(X)\right]
		\end{split}
	\end{align}
	\begin{itemize}\setlength\itemsep{2em}
		\item[]
		where
		\begin{itemize}
			\item $E(w_{j}X_{j})$: mean effect estimate for feature $j$
			\item $\phi_j$: $j$-th feature contribution
			\item $\bm{x}$: data instance
			\item $x_j$: value of feature $j$
			\item $w_j$: weight corresponding to feature $j$
			\item $p$: nr. features
		\end{itemize}
		\item<2-> predicted value minus average predicted value
	\end{itemize}
\end{frame}

\begin{frame}{Feature Contribution in General}
	\begin{itemize}
		\item \emph{Can we do the same for any type of model?}
		\begin{itemize}
			\item Model-agnostic
			\item No assumptions on features interactions
		\end{itemize}
		\item \red{Nonlinear models need a different solution!}
		\item Possible solution: \underline{Shapley value}
		\begin{itemize}
			\item Field: cooperative game theory
		\end{itemize}
	\end{itemize}
\end{frame}

\begin{frame}{Shapley Value - Intuition}
	\begin{itemize}
		\item The average marginal contribution of a feature value over all possible coalitions.
		\item Shapley value for a feature $j$: average change in prediction that a subset of features receives when the feature $j$ joins them.
	\end{itemize}
\end{frame}
\begin{frame}{Shapley Value - Feature Contribution}
\begin{equation}
	\phi_j(val_{\bm{x}})=\frac{1}{p}\sum_{S\subseteq\{1,\ldots,p\} \backslash \{j\}}\binom{p-1}{|S|}^{-1}\Bigl(val_{\bm{x}}\bigl(S\cup\{j\}\bigr)-val_{\bm{x}}(S)\Bigr)
\end{equation}
\begin{itemize}
	\item Contribution of $j$-th feature value to the prediction (\emph{payout})
	\item Normalized: weighted and summed over all possible feature combinations
	\item $j$-th feature value
	\item $val_{\bm{x}}(S)$: value of players in $S$ %\red{to explain $\bm{x}$}
	\item $S$: a subset of features used in the model (\emph{coalition})
	\item $\bm{x}$: vector of feature values of an instance to be explained
	\item $p$: nr. features
\end{itemize}
\end{frame}

\begin{frame}{Shapley Value - The Value Function - Example}
	\begin{equation}
		val_{\{1,3\}}\left(\bm{x};\hat{f}\right)=\int_{\mathbb{R}}\int_{\mathbb{R}}\hat{f}(x_{1},X_{2},x_{3},X_{4})d\mathbb{P}_{X_2X_4}-\mathbb{E}_X\left[\hat{f}(X)\right]
	\end{equation}
	\begin{itemize}\setlength\itemsep{2em}
		\item[]
		where
		\begin{itemize}
			\item $\{1,3\}$: features in coalition
			\item $p = 4$: tot. model features
			\item $\bm{x} = (x_1, \cancel{x_2}, x_3, \cancel{x_4})$: data instance
		\end{itemize}
	\end{itemize}
\end{frame}

\begin{frame}{Shapley Value - The Value Function}
	\begin{equation}
		val_S\left(\bm{x};\hat{f}\right)=\int\hat{f}(x_{1},\ldots,x_{p})d\mathbb{P}_{x\notin{}S}-\mathbb{E}_X\left[\hat{f}(X)\right]
	\end{equation}
	\begin{itemize}
		\item<1-> Payout function for coalitions of players (feature values)
		\item<1-> Predicts feature values in $S$
		\item<1-> Marginalizes over features that are not in $S$
		\item<1-> Multiple integrations for each feature that is not in $S$
		\vspace{1em}
		\item<2-> An empty coalition is worth zero
		\begin{align}\begin{split}
				val_{\{\}}\left(\bm{x};\hat{f}\right)&=
				\int\hat{f}(\bm{x})d\mathbb{P}_{\bm{x}}-\mathbb{E}_X\left[\hat{f}(X)\right]\\
				&=\mathbb{E}_X\left[\hat{f}(X)\right]-\mathbb{E}_X\left[\hat{f}(X)\right] = 0
		\end{split}\end{align}
	\end{itemize}
\end{frame}

\begin{frame}{Shapley Value - Properties}
	\begin{enumerate}
		\item Efficiency
		\item Symmetry
		\item Dummy
		\item Additivity
	\end{enumerate}
	\begin{itemize}
		\item Fair Payout
	\end{itemize}
\end{frame}
\begin{frame}{Shapley Value - Efficiency}
	\begin{equation}
		\sum\nolimits_{j=1}^p\phi_j=\hat{f}(x)-E_X(\hat{f}(X))
	\end{equation}
	\begin{itemize}
		\item Feature contributions must \emph{sum} up to prediction for $\bm{x}$ \emph{minus} average prediction
	\end{itemize}
\end{frame}
\begin{frame}{Shapley Value - Symmetry}
	If
	\begin{equation}
		val(S \cup \{j\})=val(S\cup\{k\})
	\end{equation}
	for all
	\begin{equation*}
		S\subseteq\{1,\ldots, p\} \backslash \{j,k\}
	\end{equation*}
	then
	\begin{equation*}
		\phi_j=\phi_k
	\end{equation*}
	\begin{itemize}
		\item The contribution of two feature values $j$ and $k$ should be the same, 
		\\if they contribute equally to all possible coalitions.
	\end{itemize}
\end{frame}
\begin{frame}{Shapley Value - Dummy}
	If
	\begin{equation}
		val(S\cup\{j\})=val(S)
	\end{equation}
	for all
	\begin{equation*}
		S\subseteq\{1,\ldots,p\}
	\end{equation*}
	then
	\begin{equation*}
		\phi_j=0
	\end{equation*}
	\begin{itemize}
		\item A feature $j$ that does not change the predicted value - regardless of which coalition of feature values it is added to - should have a Shapley value of 0.
	\end{itemize}
\end{frame}
\begin{frame}{Shapley Value - Additivity}
	\begin{equation}
		\phi_j+\phi_j^{+}
	\end{equation}
	\begin{itemize}
		\item Combined payouts
		\item Example: Random forest = average of many decision trees
		\begin{itemize}
			\item Prediction = average prediction in decision trees
			\item Feature contribution = average feature contribution in decision trees
		\end{itemize}
	\end{itemize}
\end{frame}

\begin{frame}{Shapley Value - Exact Estimation}
	\begin{itemize}
		\item All possible subsets (coalitions) of feature values have to be evaluated with and without the $j$-th feature.
		\item The number of possible coalitions increases exponentially as the the number of features increases.
	\end{itemize}
\end{frame}
\begin{frame}{Shapley Value - Approximation\footnotemark}
\begin{equation}
	\hat{\phi}_{j}=\frac{1}{M}\sum_{m=1}^M\left(\hat{f}(\bm{x}^{m}_{+j})-\hat{f}(\bm{x}^{m}_{-j})\right)
\end{equation}
\begin{itemize}
	\item Monte-Carlo Sampling
	\item $\hat{f}(\bm{x}^{m}_{+j})$
	\begin{itemize}
		\item prediction for a data point $\bm{x}^m$
		\item random number of features replaced by feature values from a random data point $\bm{z}^m$.
		\item uses the feature value $x^m_j$
	\end{itemize}
	\item $\hat{f}(\bm{x}^{m}_{-j})$
	\begin{itemize}
		\item like $\hat{f}(\bm{x}^{m}_{+j})$
		\item uses the random feature value $z^m_j$
	\end{itemize}
\end{itemize}
\footnotetext[3]{\citeauthor{vstrumbelj2014explaining} (\citeyear{vstrumbelj2014explaining})}
\end{frame}

\begin{frame}{Problems with Shapley Values}
\begin{itemize}
	\item Relying on training data can produce non-intuitive attributions
	\begin{itemize}
		\item Unused features can still receive attribution
	\end{itemize}
\end{itemize}
\end{frame}
\begin{frame}{Issue: Multiplicity of Shapley Values}
	\begin{itemize}
		\item Case: training data dependency
		\begin{itemize}
			\item \textbf{Sparsity} obscures model properties
			\begin{itemize}
				\item Features not relevant for the model may receive attribution
			\end{itemize}
		\end{itemize}
		\item Multiple extensions to continuous features
	\end{itemize}
\end{frame}
\begin{frame}{What-if Analysis}
	\begin{itemize}
		\item Case: intervention on the feature
		\begin{itemize}
			\item Potential issue: out-of-distribution inputs
			\item Fix: regularization
		\end{itemize}
		\item Case: marginalize the feature over the training data
		\begin{itemize}
			\item Issue: counter-intuitive explanations with sparse training data
		\end{itemize}
		\item Ideal fix: model the true feature distribution
		\begin{itemize}
			\item (typically) \red{Harder than original prediction problem!}
		\end{itemize}
	\end{itemize}
\end{frame}

\begin{frame}[fragile]{Software}
	\begin{itemize}
		\item \verb|fastshap| (R)~\cite{jethani2021fastshap}
		\item \verb|iml| (R)~\cite{molnar2018iml}
		\item \verb|breakDown| (R)~\cite{staniak2018explanations}
		\item \verb|Shapley.jl| (Julia)~\footnotemark
		\item \dots
	\end{itemize}
	\footnotetext[5]{\href{https://gitlab.com/ExpandingMan/Shapley.jl}{https://gitlab.com/ExpandingMan/Shapley.jl}}
\end{frame}

\begin{frame}{Shapley Value in Short}
	\begin{itemize}
		\item Permutation-based
		\item Model-agnostic
		\item Solid theory
		\item Full-explanation
		\begin{itemize}
			\item All the features
			\item Non sparse (proper subset of features)
		\end{itemize}
		\item Model-free
		\item Data access or generation
		\item Building block in SHAP~\cite{lundberg2017unified}
	\end{itemize}
\end{frame}

\section{References}
\begin{frame}[allowframebreaks]
\frametitle{References}
\printbibliography
\end{frame}

\end{document}